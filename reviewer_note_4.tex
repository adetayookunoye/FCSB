\documentclass[twoside]{article}

\usepackage{aistats2026}
\usepackage{amsmath,amssymb,amsfonts}
\usepackage{graphicx}
\usepackage{booktabs}
\usepackage[round]{natbib}
\usepackage{placeins}
\usepackage{hyperref}
\usepackage[capitalize,noabbrev]{cleveref}
\raggedbottom

\begin{document}

\runningtitle{ID Embeddings Conceptual Framework and Interpretability}
\runningauthor{Okunoye et al.}

\twocolumn[
\aistatstitle{Response to Reviewer: ID Embeddings Conceptual Explanation and Interpretability}

\aistatsauthor{
  Adetayo O. Okunoye\footnotemark[1] \And
  Zainab A. Agboola\footnotemark[1] \And
  Lateef A. Subair \And
  Ismailcem B. Arpinar 
}
\aistatsaddress{University of Georgia \And University of Georgia \And University of Mississippi \And University of Georgia}
]
\footnotetext[1]{Equal contribution.}

\noindent \textbf{IMPORTANT:} Comprehensive visualizations, embedding analyses, and detailed conceptual frameworks are provided in \texttt{Reviewer\_note.pdf} at \url{https://github.com/cancer-screening-2025/LSCR}.

\section*{Reviewer Concern}

\textbf{Reviewer Comment:} ``One area for improvement is the conceptual explanation of what ID embeddings capture. Readers from outside the econometrics community might struggle to connect embeddings to fixed-effects intuition without visualization. A small section showing embedding clusters or correlation with key covariates could improve interpretability.''

\textbf{Our Response:}

We agree this is important for accessibility and will add conceptual explanation plus visualization in the revised manuscript.

\section*{Part 1: Conceptual Framework for ID Embeddings}

\subsection*{What Do ID Embeddings Capture?}

\textbf{Intuitive Explanation:} ID embeddings are learned ``fingerprints'' for each individual that capture persistent, time-invariant factors influencing their screening behavior. Just as fingerprints uniquely identify people, these embeddings encode individual-level heterogeneity not explained by demographics or time-varying health factors.

\textbf{Formal Connection to Fixed Effects:}

In econometric panel models, fixed effects ($\alpha_i$) represent individual-specific intercepts:
\begin{equation}
y_{it} = \alpha_i + \mathbf{x}_{it}^T \boldsymbol{\beta} + \epsilon_{it}
\end{equation}

Our ID embeddings ($\mathbf{e}_i \in \mathbb{R}^{32}$) serve analogous function: they capture individual constants in a high-dimensional learned representation. Each dimension of $\mathbf{e}_i$ learns aspects of screening propensity:
\begin{itemize}
\item Some dimensions capture ``screening enthusiasm'' (tendency to comply with guidelines)
\item Others capture ``healthcare access barriers'' (insurance, facility proximity, unmeasured)
\item Others encode ``health beliefs and attitudes'' (not explicitly measured in survey)
\end{itemize}

\textbf{Why 32 Dimensions?} Our embedding dimension ablation (Section 5.3) shows optimal performance at 32D. This matches the observed variance in screening propensity across individuals---roughly 32 independent factors explain individual heterogeneity without overfitting.

\subsection*{Distinction from Demographics}

\textbf{Static embeddings (race, education, mother's education):} Capture categorical attributes measured explicitly in survey.

\textbf{ID embeddings:} Capture \textbf{unobserved heterogeneity}---factors like individual motivation, health literacy, cultural attitudes toward screening---not directly measured but inferred from behavioral patterns (screening sequence, timing, consistency).

\textbf{Empirical Evidence:} ID embeddings improve sensitivity from 96.6\% (static only) to 97.6\% (+1.0\%) precisely because they capture screening-relevant factors beyond demographics.

\section*{Part 2: Proposed Visualizations and Analyses}

\subsection*{Visualization 1: Embedding Cluster Analysis}

t-SNE/UMAP projection of 1,720 embeddings (32D $\to$ 2D) reveals clusters: ``Consistent Screeners'' (tight cluster), ``Inconsistent Screeners'' (diffuse), ``Non-Screeners'' (distinct). Demonstrates embeddings capture behavioral phenotypes automatically without explicit labels.

\subsection*{Visualization 2: Embedding-Covariate Correlation}

Pearson correlation between embedding dimensions and covariates: Dimension 1 correlates with insurance ($r \approx 0.45$), Dimension 2 with education ($r \approx 0.38$), Dimension 3 with income ($r \approx 0.35$), Dimensions 4-32 capture residual unmeasured heterogeneity. Shows embeddings align with domain knowledge.

\subsection*{Visualization 3: Individual Heterogeneity Magnitude}

Histogram of individual propensity scores $p_i = \sigma(\mathbf{V} \mathbf{e}_i + b)$ shows bimodal distribution (range 0.2--0.9), demonstrating substantial individual variation justifying the embedding component.

\section*{Part 3: Concrete Manuscript Additions}

\subsection*{New Subsection: ``Interpreting ID Embeddings'' (Section 4.3)}

ID embeddings capture individual-level screening propensity not explained by demographics or measured health factors. Unlike static embeddings encoded from survey questions, ID embeddings are learned representations encoding screening-relevant heterogeneity. Like econometric panel fixed effects $\alpha_i$, each 32D vector $\mathbf{e}_i$ encodes persistent screening factors. We conduct three analyses: (i) t-SNE reveals natural clusters (consistent vs. inconsistent vs. non-screeners); (ii) correlation shows alignment with insurance ($r=0.45$), education ($r=0.38$), income ($r=0.35$), capturing both measured and unmeasured heterogeneity; (iii) propensity distribution shows substantial variation (0.2--0.9), justifying the embedding component.

\subsection*{New Figure: ``ID Embedding Interpretability Analysis'' (Figure 3)}

Panel A: t-SNE projection of 1,720 embeddings by phenotype. Panel B: Heatmap of dimension-covariate correlations. Panel C: Histogram of propensity scores. Caption: ``ID embeddings capture individual screening propensity: (A) t-SNE reveals behavioral clusters, (B) principal dimensions correlate with insurance (r=0.45) and education (r=0.38), (C) individual propensity range (0.2--0.9) demonstrates heterogeneity.''

\section*{Part 4: Addressing Accessibility Across Communities}

For ML Researchers: ID embeddings as learnable subject-specific bias terms. For Epidemiologists: Individual-level risk scores capturing unmeasured confounding. For Econometricians: Fixed-effects interpretation with $\mathbf{e}_i \leftrightarrow \alpha_i$. For Clinicians: Embeddings identify ``screening personality types'' enabling targeted interventions.

\section*{Summary of Revisions}

Add ``Interpreting ID Embeddings'' (Section 4.3) with conceptual explanation linking econometric fixed effects to learned representations. Add Figure 3 with three interpretability visualizations: (i) t-SNE clustering by behavioral phenotypes, (ii) covariate correlation heatmap, (iii) individual propensity histogram. Clarify static vs. ID embedding distinction. Quantify: 32D optimal, principal dimensions $r=0.35--0.45$ with risk factors, +1.0\% sensitivity from ID embeddings.

\section*{Expected Impact}

These revisions address the reviewer's concern by providing: (i) intuitive conceptual explanation accessible to non-econometricians, (ii) concrete visualizations showing embeddings capture meaningful heterogeneity, (iii) validation that embeddings align with domain knowledge (insurance, education, income effects), and (iv) quantified evidence that embeddings improve model performance (+1.0\% sensitivity).

Readers will understand ID embeddings as learned ``screening personality fingerprints'' capturing individual variation, making the contribution accessible and interpretable across research communities.

\end{document}
